\input{Configuraciones/paquetes}

%--------------------------

\begin{document}
\input{Configuraciones/nombres}
%--------------------------

\begin{figure}[H]
	\centering
	\includegraphics[scale=0.5]{Images/1}
	\caption{Gráficas de los dos problemas}
\end{figure}

\begin{problema}
	La altura sobre la hipotenusa de un triángulo rectángulo divide el triángulo en dos triángulos directamente semejantes, cada uno de los cuales es inversamente semejante al triángulo dado. 
\end{problema}
\begin{dem}
	
	Primer caso: tenemos los triángulos $\triangle ABC$ y $\triangle ABD$. Nótese que $\angle ABC=\angle BDA=90^\circ$. Ahora, tenemos el $\angle DAB=\angle CAB$. Entonces por el criterio $AA$, $\triangle ABC\sim \triangle ABD$. Segundo caso: tenemos los triángulos $\triangle ABC$ y $\triangle DBC$. Nótese que $\angle ABC=\angle CDB=90^\circ$. Ahora, tenemos el $\angle BCA=\angle BCD$. Entonces por el criterio $AA$, $\triangle ABC\sim \triangle  DBC$. Por lo tanto, $\triangle ABC \sim \triangle ABD \sim \triangle DBC$.
\end{dem}

\begin{problema}
	Dado un triángulo rectángulo el producto de longitudes de los catetos es igual al producto de longitudes de la hipotenusa y la altura respectiva.
\end{problema}
\begin{dem}
	A probar: $AB\cdot BC= BD\cdot AC$. Tomando como referencia el problema anterior, 
	$$\frac{BD}{AB}=\frac{DC}{BC}=\frac{BC}{AC}\implies \frac{BD}{AB}=\frac{BC}{AC}\implies AB\cdot BC= BD\cdot AC $$
\end{dem}




%---------------------------
\bibliographystyle{apa}
\bibliography{referencias.bib}

\end{document}