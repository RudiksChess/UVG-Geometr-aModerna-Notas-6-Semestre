\input{Configuraciones/paquetes}

%--------------------------

\begin{document}
\input{Configuraciones/nombres}
%--------------------------

\begin{problema}
	Tangentes a las circunferencias en $A^{\prime}$ y $B$ forman ángulos iguales con la línea $A^{\prime} B $. Si estas tangentes se intersecan en $E$ el triángulo $E A^{\prime} B$ es isósceles.
	\begin{figure}[H]
		\centering
		\includegraphics[scale=0.2]{Images/1}
	\end{figure}
\end{problema}
\begin{proof}
	Nótese que por hipótesis $A',A,B,B',C,C'$ y $D,D'$ son homólogos. $\implies$ Como las circunferencias son homotéticas, entonces la tangente que pasa por $BE$ es paralela a la tangente que pasa por  $B'E'$ tal que $BE'\parallel B'E$. $\implies$ Por el teorema de paralelas y transversas $\angle A'B'E' = \angle A'B'E$. $\implies$ Por la definición de tangentes a circunferencias $\angle E'A'B' = \angle A'B'E'$. $\implies \triangle E'A'B'$ es isósceles. $\implies$ Por congruencia de los ángulos y teorema de similaridad para triángulos $\angle EA'B = \angle A'BE$. Por lo tanto, $\triangle EA'B$ es isósceles.
\end{proof}


\begin{problema}
	 La circunferencia de similitud de dos circunferencias no concéntricas es el lugar geométrico de los puntos (1) tales que las razones de sus distancias a los centros de las circunferencias son iguales a las razones entre los radios; $y$ (2) desde los cuales las dos circunferencias subtienden ángulos iguales.
	 \begin{figure}[H]
	 	\centering
	 	\includegraphics[scale=0.2]{Images/2}
	 \end{figure}
\end{problema}
\begin{proof}
	A probar: 
	\begin{enumerate}
		\item $P\in$ circunferencia de similitud $\implies PO/PO'=r/r'$. Supóngase que tenemos un punto $O''$ en el segmento de los centros, tal que el segmento $PH$ es la bisectriz de $\angle O''PO'$. $\implies$ Por el teorema de Thales, $PH \perp PK$ bisectan los $\angle$ interiores y exteriores en $P$ de $\triangle O''PO'$. $\implies$ 
		$$\frac{O''H}{HO'}= -\frac{O''K}{KO'}\quad \text{y} \quad \frac{OH}{HO'}=-\frac{OK}{KO'}\implies \frac{H}{O''}\frac{O''}{K}=\frac{HO}{OK}\implies O''=O.$$
		Por lo tanto, $$\frac{PO}{PO'}=\frac{r}{r'}.$$
		\item $PO/PO'=r/r\implies P\in$ circunferencia de similitud.  Por el teorema de la bisectriz,  $PH$ es la bisectriz del $\angle$ interior en $P$ de $\triangle OPO'$. Por otra parte, por el teorema de la bisectriz $PK$ es la bisectriz del $\angle$ exterior en $P$ de $\triangle OPO'$. Por lo tanto, $PH\perp  PK$ y $P\in $ circunferencia de similitud. 
	\end{enumerate}
\end{proof}

%---------------------------
\bibliographystyle{apa}
\bibliography{referencias.bib}

\end{document}